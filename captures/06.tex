\chapter{Deployment and Serving}

\section*{Goal}
The goal of this chapter is to deploy trained machine learning models in a reproducible, maintainable, and scalable manner.  
Deployment is the bridge between research and application: a model only becomes useful when it can be reliably consumed by other systems, users, or experiments.  
In scientific ML, the deployment stage must preserve reproducibility, numerical integrity, and version traceability.

%------------------------------------------------
\section*{Overview}
Modern MLOps platforms treat model serving as an operational service that exposes models through standardized APIs.  
Every deployed model must satisfy:
\begin{itemize}[leftmargin=1.5cm]
	\item \textbf{Reproducibility:} the deployed model must match a specific MLflow model version, container hash, and data version.
	\item \textbf{Scalability:} the service must handle concurrent prediction requests or batch jobs.
	\item \textbf{Observability:} each prediction and its inputs are logged for auditing and performance analysis.
\end{itemize}

Model serving can take different forms --- from lightweight REST APIs for experimentation to production-grade serving frameworks integrated with Kubernetes.

%------------------------------------------------
\section*{Techniques}

\subsection*{REST APIs with FastAPI or Flask}
For small-scale deployments or prototyping, a REST API is sufficient to expose model inference endpoints.  
\textbf{FastAPI} is particularly well-suited for ML because it is asynchronous, type-checked, and integrates easily with Docker and Kubernetes.

\subsection{Example FastAPI Service.}
\begin{verbatim}
from fastapi import FastAPI
import torch
from model import PINNModel

app = FastAPI(title="Inverse Heat PINN")

model = torch.jit.load("pinn_model.pt")
model.eval()

@app.get("/predict-temperature")
def predict_temperature(x: float, y: float, t: float):
    with torch.no_grad():
        inp = torch.tensor([[x, y, t]], dtype=torch.float32)
        u = model(inp).item()
        return {"u(x,y,t)": u}
\end{verbatim}

\subsection{Dockerfile for Serving.}
\begin{verbatim}
FROM python:3.11
WORKDIR /app
COPY . .
RUN pip install fastapi uvicorn torch
EXPOSE 8080
CMD ["uvicorn", "app:app", "--host", "0.0.0.0", "--port", "8080"]
\end{verbatim}

The resulting container can be built and run as:
\begin{verbatim}
docker build -t pinn-api .
docker run -p 8080:8080 pinn-api
\end{verbatim}

\subsection{Usage.}
A client can query the deployed model via:
\begin{verbatim}
curl "http://localhost:8080/predict-temperature?x=0.5&y=0.5&t=0.1"
\end{verbatim}
This returns the predicted temperature \(u(x,y,t)\).

%------------------------------------------------
\subsection*{Model Serving on Kubernetes: KServe and Seldon Core}
When scaling to production or multi-user environments, manual API services become insufficient.  
\textbf{KServe} and \textbf{Seldon Core} provide standardized Kubernetes-native serving for ML models.

\subsubsection*{KServe (KFServing)}
KServe allows deploying models directly from storage backends such as S3 or MLflow Registry without rebuilding containers.

\subsection{Example KServe Deployment.}
\begin{verbatim}
apiVersion: serving.kserve.io/v1beta1
kind: InferenceService
metadata:
  name: inverse-heat-pinn
spec:
  predictor:
    pytorch:
      storageUri: s3://mlflow-artifacts/inverse-heat-pinn/Production
      resources:
        limits:
          nvidia.com/gpu: "1"
          cpu: "2"
          memory: "8Gi"
\end{verbatim}

\subsection{Workflow.}
\begin{enumerate}
	\item Airflow or Tekton promotes the model in MLflow Registry to \texttt{Production}.
	\item ArgoCD detects a manifest change in the GitOps repository.
	\item OpenShift deploys the new \texttt{InferenceService}.
	\item KServe spins up a dedicated inference Pod with the model mounted from object storage.
\end{enumerate}

\subsection{Endpoint.}
Once deployed, KServe automatically exposes an HTTPS route such as:
\begin{verbatim}
https://inverse-heat-pinn.apps.openshift.cluster/predict
\end{verbatim}

A POST request with JSON body:
\begin{verbatim}
{
  "instances": [[0.5, 0.5, 0.1]]
}
\end{verbatim}
returns:
\begin{verbatim}
{"predictions": [0.8243]}
\end{verbatim}

\subsubsection*{Seldon Core}
\textbf{Seldon Core} is an alternative to KServe with richer model routing, A/B testing, and custom graph definitions.

\subsection{Example Seldon Deployment.}
\begin{verbatim}
apiVersion: machinelearning.seldon.io/v1
kind: SeldonDeployment
metadata:
  name: pinn-deployment
spec:
  predictors:
  - name: default
    replicas: 2
    graph:
      name: model
      implementation: PYTORCH_SERVER
      modelUri: s3://mlflow-artifacts/inverse-heat-pinn/Production
    componentSpecs:
    - spec:
        containers:
        - name: model
          resources:
            limits:
              nvidia.com/gpu: "1"
\end{verbatim}

Seldon integrates naturally with OpenShift’s service mesh (Istio) for routing and traffic control between different model versions.

%------------------------------------------------
\section*{Continuous Deployment (CI/CD)}
To maintain reliability, model deployment must be integrated into the CI/CD process.

\subsection{Pipeline Example (GitLab CI/CD).}
\begin{verbatim}
stages:
  - build
  - deploy

build:
  stage: build
  script:
    - docker build -t registry/pinn-api:$CI_COMMIT_SHA .
    - docker push registry/pinn-api:$CI_COMMIT_SHA

deploy:
  stage: deploy
  script:
    - kubectl apply -f manifests/kserve-inverse-heat.yaml
  only:
    - main
\end{verbatim}

\subsection{Alternative Tools.}
\begin{itemize}[leftmargin=1.5cm]
	\item \textbf{Jenkins:} suitable for on-prem clusters and legacy CI/CD pipelines.
	\item \textbf{Argo CD:} declarative GitOps tool continuously reconciling the live cluster state with Git manifests.
\end{itemize}

\subsection{Integration with OpenShift.}
On OpenShift, Tekton and ArgoCD provide native CI/CD and GitOps.  
Models promoted to the MLflow Registry automatically trigger new pipeline runs that update the deployed inference service.

%------------------------------------------------
\section*{Deployment Strategies}

\subsection*{Blue-Green Deployment}
Two identical environments (\texttt{blue} and \texttt{green}) exist concurrently:
\begin{itemize}
	\item The \texttt{blue} environment serves current production traffic.
	\item The \texttt{green} environment deploys the new model.
\end{itemize}
Once validation passes, traffic is switched from blue to green.  
Rollback is instantaneous by redirecting traffic back to the old version.

\subsection{Implementation in OpenShift.}
Using OpenShift Routes:
\begin{verbatim}
oc set route-backends inverse-heat-pinn blue=0 green=100
\end{verbatim}
can shift traffic between model versions atomically.

\subsection*{Canary Deployment}
Instead of a full switch, traffic is gradually shifted to the new model version to monitor stability:
\begin{verbatim}
oc set route-backends inverse-heat-pinn blue=90 green=10
\end{verbatim}
Monitoring tools (Prometheus, Grafana, Evidently) track error rates, latency, and drift metrics.  
If no degradation is detected, traffic is progressively increased until the new model fully replaces the old one.

%------------------------------------------------
\section*{Operational Integration for the PINN Project}
\subsection{Scenario.}
After training and validation, a new PINN model for estimating \(a(x,y)\) and \(f(x,y,t)\) is promoted in MLflow and deployed automatically.

\subsection{Pipeline Steps.}
\begin{enumerate}
	\item \textbf{Airflow or Tekton} detects new MLflow model promotion.
	\item \textbf{ArgoCD} synchronizes the updated KServe manifest in Git.
	\item \textbf{OpenShift} deploys the new inference service with GPU scheduling.
	\item \textbf{Prometheus/Grafana} monitor inference latency and physical consistency.
	\item \textbf{Evidently AI} analyzes prediction drift; significant drift triggers retraining.
\end{enumerate}

\subsection{API Specification for PINN Inference.}
\begin{verbatim}
/predict-temperature(x, y, t)
  Method: GET
  Input:  float x, y, t
  Output: float u(x,y,t)
\end{verbatim}

Example query:
\begin{verbatim}
curl "https://inverse-heat-pinn.apps.cluster/predict-temperature?x=0.5&y=0.5&t=0.1"
\end{verbatim}

\subsection{Response Example.}
\begin{verbatim}
{
  "u(x,y,t)": 0.8243,
  "model_version": "1.3.2",
  "data_version": "dvc-8fa23b",
  "timestamp": "2025-10-20T15:22:17Z"
}
\end{verbatim}

The response includes full lineage metadata, enabling complete traceability and reproducibility of scientific predictions.

%------------------------------------------------
\section*{Summary}
Model deployment is the culmination of the MLOps lifecycle --- transforming reproducible research artifacts into operational services.  
For the inverse heat PINN project:
\begin{itemize}[leftmargin=1.5cm]
	\item FastAPI enables quick experimentation and REST access.
	\item KServe and Seldon Core deliver scalable inference on Kubernetes/OpenShift.
	\item CI/CD systems (GitLab, Jenkins, ArgoCD) automate builds and rollouts.
	\item Blue-Green and Canary strategies minimize downtime and risk.
	\item Monitoring (Prometheus, Grafana, Evidently) ensures reliability and retraining triggers.
\end{itemize}

Through these mechanisms, scientific models evolve into robust, production-grade services that remain explainable, auditable, and continuously improvable.
