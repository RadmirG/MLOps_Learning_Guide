\chapter{Inhalte der vollen Masterarbeit}
\label{app:cdrom}

\section{Sourcecode} 
Der gesamte Sourcecode der Masterarbeit ist in der GitHub Repository: \href{https://github.com/RadmirG/Master-Arbeit}{\textbf{Radmir Gesler Master Arbeit}} zu finden


\section{PDF-Dateien}
\begin{itemize}
	\item \href{https://github.com/RadmirG/Master-Arbeit/blob/master/MA_RadmirGesler.pdf}{MA\_RadmirGesler.pdf} - gesamte Masterarbeit im PDF-Format 
\end{itemize}

\section{Python Code}

\begin{itemize}
	
	\item Den Sourcecode der Implementierung findet man im Verzeichnis \href{https://github.com/RadmirG/Master-Arbeit/tree/master/code/InverseHeatSolver}{/code/InverseHeatSolver}. Dabei sind unten aufgeführte Inhalte für gesamte Arbeit wesentlich.

    \begin{verbatim}
|-- FEniCS_scripts
    |-- direct_1D_fenics.eps
    |-- FEniCS_1D_td.py
    |-- FEniCS_1D_ti.py
    |-- FEniCS_2D_td.py
    |-- FEniCS_Dockerfile
    |-- ForwardHeatSolver.py
    |-- test_cases_FEniCS.py
|-- solver
    |-- History.py
    |-- Interpolator.py
    |-- InverseHeatSolver.py
    |-- ModelComposer.py
    |-- PdeMinimizer.py
    |-- PdeMinimizerDeepXde.py
    |-- Visualizer.py
|-- cases_and_plots.ipynb
|-- dde_forward_1D_ti.py
|-- dde_forward_2D_ti.py
|-- functions.py
|-- requirements.txt
|-- use_cases.py
|-- use_cases_deep_xde.py
    \end{verbatim}
    
Der Ordner \texttt{FEniCS\_scripts} enthält Skripte für die FEM Berechnung des Vorwärtsproblems mit FEniCS. Der Ordner \texttt{solver} enthält alle Objekte zur Lösung des inversen Problems.

\end{itemize}